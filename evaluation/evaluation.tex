\chapter{Experiments and Results}\label{Evalutation}

This chapter introduces the results obtained from the steps detailed in the previous chapter. It begins with a discussion and an overview of the corpus collected. Then, a section presenting sentiment analysis as well as the frequency of term mentions within the previously created specialist dictionaries (Chapter \ref{Data Collection}). Finally, a statistical analysis will be performed such as to identify any potential relationship between the presented variables. 

\section{Data Collection}

From the original 369,569 articles downloaded using the approach detailed in Chapter \ref{Collecting Data}, a total of 305,454 were loaded into the database after date and source filtering was performed, removing a total of 17.34\% of articles. Table \ref{tab: corpus size} includes key information with regards to  the size of the corpus. The selected period makes for a total of 450 days, the average daily article count is then of 679 articles per day. Table \ref{tab:core stat source} and Table \ref{tab:stat source} include core statistics relating to sources, on average individual sources contain 1351.57 articles with a very high standard deviation of 3220.96 which shows that the data is heavily dispersed, as we can see looking at the minimum and maximum article count per source, 1 and 30,767 respectively, we can clearly see how that can be the case. Furthermore, as previously explained (Chapter \ref{Statistical Methods}), the skewness measures the distribution of articles, 5.61 is considered a high value which signifies that the data is heavily skewed to the right, meaning that the right side of the distribution is longer. This is confirmed by the high kurtosis value of 40.31, showing a high amount of outliers in our data distribution, confirming the above analysis.

Table \ref{tab:stat source} displays metrics about the thirty largest sources, it is found that while five sources have over 10,000 articles, it quickly drops with the last few sources showing article counts closer to 3,000 which is much closer to the observed mean.

Looking at the largest source, \emph{Agence France Presse} (AFP), its details have been previously discussed (Chapter \ref{chap:French Press}). The following sources are part of the largest French news organisations so it comes as no surprise that their article count is very high.

Analysing the sources with the least amount of articles, it is found that 158 sources have less than 1000 articles and 104 have less than 200. Considering that the total number of selected sources is 226, the majority are found to contain a number of articles significantly lower than the mean as the counts are quite dispersed. Digging deeper, it is found that some sources appear under different names, for instance, Table \ref{tab:source repetition} shows an example of one source appearing under many different regional editions. This is occurs in many times within the corpus which explains how the sources are so spread out. 

\begin{table}[H]
\centering
\begin{tabular}{lll}
\toprule
Sources & Articles & Words \\ \hline
226 & 305,454 & 167,800,491 \\
\bottomrule
\end{tabular}
\caption{Corpus Size}
\label{tab: corpus size}
\end{table}

\begin{table}[H]
\centering
\begin{tabular}{@{}lllllll@{}}
\toprule
 & Mean & Std Dev & Skewness & Kurtosis & Minimum & Maximum \\ \midrule
Articles per Source & 1352 & 3221 & 6 & 40 & 1 & 30,767 \\ \bottomrule
\end{tabular}
\caption{Article Statistics Overview}
\label{tab:core stat source}
\end{table}

\begin{table}[H]
\caption{Example of Source Repetition}
\label{tab:source repetition}
\centering
\begin{tabular}{l}
\toprule
\emph{20 Minutes} \\
\emph{20 Minutes Bordeaux} \\
\emph{20 Minutes Lille} \\
\emph{20 Minutes Lyon} \\
\emph{20 Minutes Marseille} \\
\emph{20 Minutes Montpellier} \\
\emph{20 Minutes Nantes} \\
\emph{20 Minutes Nice} \\
\emph{20 Minutes Paris} \\
\emph{20 Minutes Rennes} \\
\emph{20 Minutes Strasbourg} \\
\emph{20 Minutes Toulouse} \\
\bottomrule
\end{tabular} \\[0.2cm]
Some sources appear under different names in the corpus, this causes for the data as observed in Table \ref{tab:core stat source} to be spread out.
\end{table}

\begin{table}[H]
\caption{Article Breakdown Per Source}
\label{tab:stat source}
\resizebox{\textwidth}{!}{%
\begin{tabular}{@{}|crrr|@{}}
\toprule
\textbf{Source (First 30)} & \multicolumn{1}{c}{\textbf{Article Count}} & \multicolumn{1}{c}{\textbf{Word Count}} & \multicolumn{1}{c|}{\textbf{Mean Words}} \\ \midrule
\emph{Agence France Presse} & 30767 & 14808082 & 481 \\
\emph{Lefigaro.fr} & 21978 & 12111314 & 551 \\
\emph{Le Dauphiné Libéré} & 17888 & 6334269 & 354 \\
\emph{Le Progrès} & 11424 & 4345194 & 380 \\
\emph{Challenges.fr} & 11139 & 6740282 & 605 \\
\emph{La Depeche du Midi} & 9687 & 3750146 & 387 \\
\emph{LesEchos.fr} & 8801 & 7457331 & 847 \\
\emph{La Nouvelle République du Centre Ouest} & 7180 & 2504550 & 349 \\
\emph{Le Télégramme} & 6853 & 2140967 & 312 \\
\emph{La Montagne} & 5625 & 2042911 & 363 \\
\emph{L'Est Républicain} & 5529 & 1960041 & 355 \\
\emph{Les Echos} & 5212 & 3291085 & 631 \\
\emph{Vosges Matin} & 4754 & 1722946 & 362 \\
\emph{MIDI LIBRE} & 4719 & 1717583 & 364 \\
\emph{Le Figaro} & 4712 & 3188302 & 677 \\
\emph{Courrier de l'Ouest} & 4645 & 1573221 & 339 \\
\emph{Le Figaro Économie} & 4556 & 3094663 & 679 \\
\emph{Sciences et Avenir.fr} & 4539 & 3316758 & 731 \\
\emph{L'Obs} & 3649 & 2798011 & 767 \\
\emph{Nice Matin} & 3453 & 2744343 & 795 \\
\emph{Le Courrier Picard} & 3356 & 1306574 & 389 \\
\emph{Le Républicain Lorrain} & 3346 & 1222386 & 365 \\
\emph{L'INDEPENDANT} & 3276 & 1100647 & 336 \\
\emph{La Tribune} & 3049 & 2782505 & 913 \\
\emph{LePoint.fr} & 2923 & 2909518 & 995 \\
\emph{L'Est-eclair - Liberation Champagne} & 2914 & 1100477 & 378 \\
\emph{Le Petit Journal} & 2860 & 830362 & 290 \\
\emph{FRANCE 24 (French)} & 2806 & 1733556 & 618 \\
\emph{Var Matin} & 2761 & 1996749 & 723 \\
\emph{La Tribune.fr} & 2732 & 2405697 & 881 \\
\emph{Le Berry Républicain} & 2718 & 1067047 & 393 \\ \bottomrule
\end{tabular}%
}
\end{table}

\section{Sentiment Analysis}

\subsection{Polarity Analysis}

Looking at the scores for polarity on the corpus as shown in Table \ref{tab:dict stats}, means and standard deviations for the three dictionaries are found to be quite different, that is simply because they each use a specific weighting system. The following results will focus on the output of FEEL (Chapter \ref{chap: feel}) as it also includes sentiment for a set of emotions such as fear, joy and more and, like the specialised dictionaries, its weighing system simply count word occurrences. An overview of the sentiment scores can be found in Table \ref{tab:feel stats}, given that FEEL counts the number of matched terms for each category, scores have been converted to percentages of the articles lengths such as to make these numbers easier to understand. 

We find that positivity and negativity occur the most often in the corpus (23\% and 6.2\% on average respectively) with other emotions representing a range between 1.12\% to 4.63\% of all terms on average. Positivity, negativity and fear are considered "relatively symmetric" looking at the skewness value, on the other hand, joy, sadness, anger, surprise and disgust are found to be highly skewed. The kurtosis measure for most of these measures are found to be relatively close to the normal distribution, with the exception of joy (5.32) and surprise (9.16) which have high kurtosis values indicating that these distributions have a large amount of outliers.

\begin{table}[H]
\centering
\begin{tabular}{@{}|c|llll|@{}}
\toprule
 & \multicolumn{1}{c}{\textbf{Mean}} & \multicolumn{1}{c}{\textbf{Std Dev}} & \multicolumn{1}{c}{\textbf{Min}} & \multicolumn{1}{c|}{\textbf{Max}} \\ \midrule
\textbf{FEEL Positive} & 0.230 & 0.034 & 0.000 & 0.500 \\
\textbf{FEEL Negative} & 0.062 & 0.022 & 0.000 & 0.309 \\
\textbf{Polarimots Positive} & 0.014 & 0.007 & 0.000 & 0.102 \\
\textbf{Polarimots Negative} & 0.004 & 0.004 & 0.000 & 0.105 \\
\textbf{Diko Positive} & 0.395 & 0.068 & 0.000 & 2.498 \\
\textbf{Diko Negative} & 0.144 & 0.084 & 0.000 & 1.647 \\ \bottomrule
\end{tabular}
\caption{Dictionaries Sentiment Comparison}
Values were converted to be relative to the corpus size and absolute.\\Values represent individual articles with no grouping.
\label{tab:dict stats}
\end{table}

\begin{table}[H]
\centering
\begin{tabular}{@{}ccccc@{}}
\toprule
         & Mean    & Std Dev & Skewness & Kurtosis \\ \midrule
Positive & 22.98\% & 3.36\%  & -0.17  & 2.55   \\
Negative & 6.18\%  & 2.18\%  & 0.50   & 1.30   \\
Joy      & 1.12\%  & 0.75\%  & 1.48   & 5.32   \\
Fear     & 4.63\%  & 1.84\%  & 0.63   & 1.64   \\
Sadness  & 3.81\%  & 1.72\%  & 0.92   & 2.63   \\
Anger    & 2.43\%  & 1.32\%  & 1.10   & 3.34   \\
Surprise & 2.26\%  & 1.14\%  & 1.31   & 9.16   \\
Disgust  & 1.81\%  & 1.12\%  & 1.20   & 3.61   \\
\bottomrule
\end{tabular}
\caption{FEEL Sentiment Overview}
\label{tab:feel stats}
\emph{Values represent individual articles with no grouping.}
\end{table}

\subsection{Virus Mentions}

The virus dictionary is the largest of all three so it is no surprise that within the dictionary, the mean rate of mentions per word is of 1.39\% and the standard deviation is of 3.18\%. The mean and standard deviation show that a select set of words represent most of the mentions. This is confirmed in Table \ref{tab:virus freqs} which shows the fifteen most mentioned terms over the corpus, clearly the first seven terms make up most of the mentions. In total, words of this dictionary have been found 2,250,684 times within the corpus which makes for a total of 1.34\% of all words within the dictionary.

\begin{table}[H]
\caption{Virus Dictionary Analysis}
\label{tab:virus freqs}
\centering
\begin{tabular}{@{}cc@{}}
\toprule
Word      & Proportion \\ \midrule
covid-19      & 19.45\% \\
coronavirus   & 10.92\% \\
pandémie      & 9.01\%  \\
confinement   & 8.43\%  \\
virus         & 7.98\%  \\
épidémie      & 7.96\%  \\
masque        & 6.62\%  \\
contamination & 2.68\%  \\
réanimation   & 2.42\%  \\
restriction   & 1.94\%  \\
dépistage     & 1.59\%  \\
infection     & 1.57\%  \\
déconfinement & 1.53\%  \\
symptôme      & 1.49\%  \\
contaminer    & 1.39\%  \\ \bottomrule
\end{tabular}
\end{table}

\subsection{Vaccine Mentions}

The vaccine dictionary has matched a total of 510,088 words in the corpus, in percentages this makes 0.30\% of the corpus. Within the dictionary, the mean rate of mentions is of 3.70\% and the standard deviation of 6.88\% hence the data is quite spread out. Once again, as shown in Table \ref{tab:vaccine freqs}, the first four words make up for 67\% of the entire dictionary mentions.

\begin{table}[H]
\caption{Vaccine Dictionary Analysis}
\label{tab:vaccine freqs}
\centering
\begin{tabular}{@{}cc@{}}
\toprule
Word      & Proportion \\ \midrule
vaccin      & 33.26\% \\
vaccination & 15.38\% \\
dose        & 10.10\% \\
vacciner    & 8.33\%  \\
pfizer      & 4.54\%  \\
astrazeneca & 4.19\%  \\
injection   & 3.11\%  \\
biontech    & 2.84\%  \\
anticorps   & 2.75\%  \\
moderna     & 2.41\%  \\
vaccinal    & 2.13\%  \\
immunité    & 2.13\%  \\
sanofi      & 1.77\%  \\
administrer & 1.67\%  \\
arn         & 1.05\%  \\ \bottomrule
\end{tabular}
\end{table}

\subsection{Death Mentions}

The death dictionary shows similar metrics to the vaccine dictionary. Its words were found a total of 527,011 times in the corpus, making for 0.31\% of the total corpus. The mean rate of mentions within the dictionary is of 1.56\% and the standard deviation of 4.52\%, once again showing a high value. Similarly, the rate of mentions is quite disproportionate, as shown in Table \ref{tab:death freqs}, the first four words make up for 73.77\% of the entire dictionary mentions.

\begin{table}[H]
\caption{Death Dictionary Analysis}
\label{tab:death freqs}
\centering
\begin{tabular}{@{}cc@{}}
\toprule
Word      & Proportion \\ \midrule
hôpital     & 22.21\% \\
mort        & 18.87\% \\
décès       & 17.32\% \\
maladie     & 15.37\% \\
victime     & 4.43\%  \\
décéder     & 3.87\%  \\
mourir      & 3.30\%  \\
tuer        & 2.25\%  \\
disparaître & 1.80\%  \\
mortel      & 0.95\%  \\
tombe       & 0.82\%  \\
disparition & 0.74\%  \\
cimetière   & 0.58\%  \\
exécution   & 0.56\%  \\
éteindre    & 0.55\%  \\ \bottomrule
\end{tabular}
\end{table}

\section{Statistical Analysis}

\subsection{Correlation}

Tables \ref{tab: correlation 1/2} and  \ref{tab: correlation 2/2} show an overview of the correlations found between the computed sentiment variables as well as France's COVID-19 epidemiology data provided by the French government\footnote{\url{https://www.data.gouv.fr/en/datasets/donnees-relatives-a-lepidemie-de-covid-19-en-france-vue-densemble/}} (Chapter \ref{chap: output aggregating}). Bold values are statistically significant with P < 0.05.

The number of daily articles is found to be positively correlated with the death count (0.50) as well as negativity, fear, sadness, anger and surprise, but is found to be negatively correlated with both virus and death sentiment. The number of new covid cases is found to be positively correlated with the number of daily death, vaccinations and new cases but is found to be negatively correlated with both virus and death sentiment, as well as joy, fear, surprise and disgust. Surprisingly, the number of daily deaths is not correlated with death sentiment. Instead it is found to be positively correlated with the vaccination rate and sentiment, as well as the average article length. It is also found to be negatively correlated with virus sentiment, joy and disgust. The daily vaccination rate is unsurprisingly found to be positively correlated (0.58) with vaccine sentiment, however, it is also negatively correlated with virus and death sentiment, as well as all of the FEEL recorded sentiments. It is surprising that positive terms are positively correlated with negative terms (0.62), in fact it is found that most of the FEEL recorded emotions are positively correlated. To finish, vaccine sentiment is positively correlated with death sentiment (0.63).

\begin{table}[H]
\centering
\resizebox{\textwidth}{!}{%
\begin{tabular}{|c|cccccccc|}
\hline
\multicolumn{1}{|l|}{} & ArticleCount & Cases & Death & Doses & VaccineDict & VirusDict & DeathDict & Length \\ \hline
ArticleCount &  &  &  &  &  &  &  &  \\
Cases &  &  &  &  &  &  &  &  \\
Death & 0.50 & 0.43 &  &  &  &  &  &  \\
Doses &  & 0.45 & 0.28 &  &  &  &  &  \\
VaccineDict &  & 0.44 & 0.48 & 0.58 &  &  &  &  \\
VirusDict & -0.21 & -0.29 & -0.15 & -0.26 & -0.19 &  &  &  \\
DeathDict & -0.13 & -0.28 &  & -0.22 & -0.25 & 0.63 &  &  \\
Length & 0.99 &  & 0.48 &  &  & -0.17 &  &  \\
Pos &  &  &  & -0.12 & 0.18 & 0.12 &  &  \\
Neg & 0.18 &  &  & -0.19 &  & 0.20 & 0.10 & 0.17 \\
Joy &  & -0.12 & -0.15 & -0.16 &  & 0.18 &  &  \\
Fear & 0.14 & -0.11 &  & -0.21 &  & 0.10 & 0.21 & 0.14 \\
Sadness & 0.18 & -0.14 &  & -0.29 & -0.15 &  & 0.13 & 0.18 \\
Anger & 0.35 &  &  & -0.16 &  & -0.16 & -0.10 & 0.33 \\
Surprise & 0.24 & -0.12 &  & -0.30 & -0.16 &  &  & 0.22 \\
Disgust &  & -0.19 & -0.10 & -0.18 &  & 0.26 & 0.16 &  \\ \hline
\end{tabular}%
}
\caption{Correlation Matrix of Sentiment and Government COVID-19 Epidemiology (Table 1/2)}
\label{tab: correlation 1/2}
Only statistically significant correlations with p < 0.05 are shown.
\end{table}


\begin{table}[H]
\centering
\resizebox{\textwidth}{!}{%
\begin{tabular}{|c|cccccccc|}
\hline
\multicolumn{1}{|l|}{} & Pos & Neg & Joy & Fear & Sadness & Anger & Surprise & Disgust \\ \hline
Neg & 0.62 &  &  &  &  &  &  &  \\
Joy & 0.50 &  &  &  &  &  &  &  \\
Fear & 0.59 & 0.88 &  &  &  &  &  &  \\
Sadness & 0.64 & 0.71 & 0.18 & 0.76 &  &  &  &  \\
Anger & 0.56 & 0.72 &  & 0.69 & 0.64 &  &  &  \\
Surprise & 0.61 & 0.55 & 0.26 & 0.51 & 0.69 & 0.71 &  &  \\
Disgust & 0.23 & 0.70 &  & 0.56 & 0.22 & 0.31 & 0.19 &  \\ \hline
\end{tabular}%
}
\caption{Correlation Matrix of Sentiment and Government COVID-19 Epidemiology (Table 2/2)}
Only statistically significant correlations with p < 0.05 are shown.
\label{tab: correlation 2/2}
\end{table}

\subsection{VAR Analysis}

A VAR analysis was performed on various models as defined previously (Chapter \ref{chap: stat analysis}). Models are built incrementally by adding one variable at a time and evaluating the results.

Firstly, variables are tested using the Augmented Dickey–Fuller test (ADF). This test checks that no unit root is present in each variable, hence making sure that the variables are stationary. In the case that the ADF test fails, variables are tested using the Engle Granger cointegration test. This test always failed in this experiment however had the test succeeded a vector error correction model (VEC) would have be used instead of a vector autoregression (VAR) model. Variables which failed the ADF test and the cointegration test were first-order differentiated. The selection of an optimal lag length is then performed by the use of the Akaike Information Criterion (AIC), Bayesian Information Criterion (BIC) and the Hannan-Quinn criterion (HQC). Finally, the resulting variables were added to the VAR model incrementally such as to quantify the differences each variable makes to the obtained model.

\subsubsection{Death Sentiment}

The variables for the number of deaths and cases failed both the ADF and cointegration tests. Their first-order difference has been used instead in this model as a result.

Table \ref{tab:model 1 lag} shows the results obtained for lag selection. We find that the three selection metrics all selected different length. BIC selected a lag of 1 which was ignored, secondly AIC selected lag 9 and HQC selected a lag of 7. Lag 9 was used in this instance as it was found to obtain superior results through manual testing.

\begin{table}[H]
\centering
\begin{tabular}{@{}llll@{}}
\toprule
\multicolumn{1}{c}{\textbf{lags}} & \multicolumn{1}{c}{\textbf{AIC}} & \multicolumn{1}{c}{\textbf{BIC}} & \multicolumn{1}{c}{\textbf{HQC}} \\ \midrule
1 & -13.71 & \textbf{-13.29*} & -13.54 \\
2 & -13.93 & -13.15 & -13.62 \\
3 & -14.07 & -12.93 & -13.62 \\
4 & -14.21 & -12.72 & -13.62 \\
5 & -14.31 & -12.44 & -13.57 \\
6 & -14.54 & -12.30 & -13.65 \\
7 & -15.00 & -12.40 & \textbf{-13.97*} \\
8 & -15.09 & -12.13 & -13.92 \\
9 & \textbf{-15.17*} & -11.85 & -13.86 \\
10 & -15.12 & -11.44 & -13.66 \\ \bottomrule
\end{tabular}
\caption{Model 1 Lag Selection}
\label{tab:model 1 lag}
\end{table}

Finally, Table \ref{tab:model 1 var} displays the VAR results regressing each variable against Death Sentiment and Table \ref{tab:model 1 causality} performs a Granger causality test on all variables.

The death sentiment is found to have a positive relationship with lagged values of itself, at lag 1, 3, 6, 7 and 9 with a largest coefficient of 0.530. COVID-19 recorded daily deaths has been found to have a very minor positive effect for most lags (in the order of $10^{-6}$). Virus sentiment has been found to have a minor negative relationship on lags 5, 6, 7 and 8 with the largest coefficient on lag 6 with a value of $-0.052$. The number of daily recorded cases has been found to not improve the model (as seen by a small decrease in adjusted $R^{2}$) so it was omitted from further models. Negative sentiment has only been found to have a small positive relationship on lag 5 and 8 and has slightly improved the adjusted $R^2$ value. To finish, the fear sentiment has a negative relationship with a single small significant value of small order (-0.085) on lag 5.

It is worth noting that the Durbin-Watson values for this model are found to vary between 1.854 to 1.922 which is slightly below 2 thus showing signs of small positive autocorrelation. 

As seen in Table \ref{tab:model 1 causality}, the observed P values are found to be extremely small thus always showing a bidirectional causality.

\begin{longtable}[c]{@{}lrrrrrr@{}}
\caption{Model 1 VAR Analysis}
\label{tab:model 1 var}\\
\toprule
\multicolumn{1}{c}{\textbf{}} & \multicolumn{1}{c}{\textbf{1.1}} & \multicolumn{1}{c}{\textbf{1.2}} & \multicolumn{1}{c}{\textbf{1.3}} & \multicolumn{1}{c}{\textbf{1.4}} & \multicolumn{1}{c}{\textbf{1.5}} &
\multicolumn{1}{c}{\textbf{1.6}} \\* \midrule
\endfirsthead
%
\multicolumn{7}{c}%
{{\bfseries Table \thetable\ continued from previous page}} \\
\toprule
\multicolumn{1}{c}{\textbf{}} & \multicolumn{1}{c}{\textbf{1.1}} & \multicolumn{1}{c}{\textbf{1.2}} & \multicolumn{1}{c}{\textbf{1.3}} & \multicolumn{1}{c}{\textbf{1.4}} & \multicolumn{1}{c}{\textbf{1.5}} &
\multicolumn{1}{c}{\textbf{1.6}} \\* \midrule
\endhead
%
\bottomrule
\endfoot
%
\endlastfoot
%
const & \textbf{0.000} & \textbf{0.000} & 0.000 & 0.000 & 0.001 & 0.000 \\
DeathDict\_1 & \textbf{0.588} & \textbf{0.571} & \textbf{0.593} & \textbf{0.593} & \textbf{0.550} & \textbf{0.530} \\
DeathDict\_2 & 0.020 & -0.066 & -0.074 & -0.080 & -0.051 & -0.044 \\
DeathDict\_3 & \textbf{0.101} & \textbf{0.182} & \textbf{0.238} & \textbf{0.241} & \textbf{0.208} & \textbf{0.192} \\
DeathDict\_4 & 0.022 & 0.044 & \textbf{-0.102} & \textbf{-0.098} & -0.058 & -0.046 \\
DeathDict\_5 & -0.085 & \textbf{-0.115} & -0.010 & -0.014 & -0.008 & 0.039 \\
DeathDict\_6 & \textbf{0.082} & \textbf{0.166} & \textbf{0.148} & \textbf{0.153} & 0.083 & \textbf{0.116} \\
DeathDict\_7 & \textbf{0.375} & \textbf{0.181} & \textbf{0.193} & \textbf{0.197} & \textbf{0.223} & \textbf{0.215} \\
DeathDict\_8 & 0.009 & 0.068 & 0.064 & 0.055 & 0.071 & 0.061 \\
DeathDict\_9 & \textbf{-0.197} & \textbf{-0.100} & \textbf{-0.190} & \textbf{-0.189} & \textbf{-0.165} & \textbf{-0.162} \\
Deaths\_1 &  & \textbf{1.69E-06} & \textbf{1.73E-06} & \textbf{1.34E-06} & \textbf{1.80E-06} & \textbf{1.69E-06} \\
Deaths\_2 &  & \textbf{1.80E-06} & \textbf{1.81E-06} & \textbf{1.63E-06} & \textbf{2.02E-06} & \textbf{1.94E-06} \\
Deaths\_3 &  & \textbf{1.10E-06} & \textbf{1.14E-06} & 7.29E-07 & \textbf{1.24E-06} & \textbf{1.32E-06} \\
Deaths\_4 &  & \textbf{2.17E-06} & \textbf{1.85E-06} & \textbf{2.08E-06} & \textbf{2.20E-06} & \textbf{2.32E-06} \\
Deaths\_5 &  & \textbf{2.27E-06} & \textbf{2.02E-06} & \textbf{2.09E-06} & \textbf{2.30E-06} & \textbf{2.51E-06} \\
Deaths\_6 &  & \textbf{2.04E-06} & \textbf{1.57E-06} & \textbf{1.72E-06} & \textbf{1.64E-06} & \textbf{1.94E-06} \\
Deaths\_7 &  & -2.64E-07 & -6.91E-07 & -5.97E-07 & -4.77E-07 & -1.52E-07 \\
Deaths\_8 &  & -8.08E-07 & -1.05E-06 & -1.18E-06 & -9.25E-07 & -6.35E-07 \\
Deaths\_9 &  & -1.16E-06 & \textbf{-1.48E-06} & -1.12E-06 & \textbf{-1.48E-06} & \textbf{-1.32E-06} \\
VirusDict\_1 &  &  & 0.019 & 0.022 & \textbf{0.044} & 0.029 \\
VirusDict\_2 &  &  & 0.019 & 0.020 & 0.017 & 0.007 \\
VirusDict\_3 &  &  & 0.006 & 0.005 & -0.002 & 0.008 \\
VirusDict\_4 &  &  & 0.005 & 0.004 & -0.001 & 0.000 \\
VirusDict\_5 &  &  & \textbf{-0.025} & \textbf{-0.026} & -0.023 & \textbf{-0.033} \\
VirusDict\_6 &  &  & \textbf{-0.060} & \textbf{-0.060} & \textbf{-0.053} & \textbf{-0.052} \\
VirusDict\_7 &  &  & \textbf{0.043} & \textbf{0.043} & \textbf{0.041} & \textbf{0.047} \\
VirusDict\_8 &  &  & -0.008 & -0.008 & -0.004 & -0.002 \\
VirusDict\_9 &  &  & \textbf{0.028} & \textbf{0.028} & 0.010 & 0.014 \\
Cases\_1 &  &  &  & -6.27E-09 &  &  \\
Cases\_2 &  &  &  & -3.95E-09 &  &  \\
Cases\_3 &  &  &  & -7.58E-09 &  &  \\
Cases\_4 &  &  &  & 1.09E-09 &  &  \\
Cases\_5 &  &  &  & -1.05E-09 &  &  \\
Cases\_6 &  &  &  & 3.67E-09 &  &  \\
Cases\_7 &  &  &  & 7.59E-09 &  &  \\
Cases\_8 &  &  &  & 2.53E-09 &  &  \\
Cases\_9 &  &  &  & \textbf{1.15E-08} &  &  \\
Neg\_1 &  &  &  &  & -0.006 & -0.023 \\
Neg\_2 &  &  &  &  & -0.010 & -0.027 \\
Neg\_3 &  &  &  &  & -0.008 & -0.033 \\
Neg\_4 &  &  &  &  & 0.001 & 0.034 \\
Neg\_5 &  &  &  &  & -0.006 & \textbf{0.046} \\
Neg\_6 &  &  &  &  & -0.009 & 0.003 \\
Neg\_7 &  &  &  &  & -0.017 & -0.022 \\
Neg\_8 &  &  &  &  & \textbf{0.036} & \textbf{0.057} \\
Neg\_9 &  &  &  &  & 0.006 & 0.012 \\
Fear\_1 &  &  &  &  &  & 0.024 \\
Fear\_2 &  &  &  &  &  & 0.046 \\
Fear\_3 &  &  &  &  &  & 0.058 \\
Fear\_4 &  &  &  &  &  & -0.060 \\
Fear\_5 &  &  &  &  &  & \textbf{-0.085} \\
Fear\_6 &  &  &  &  &  & -0.016 \\
Fear\_7 &  &  &  &  &  & 0.006 \\
Fear\_8 &  &  &  &  &  & -0.031 \\
Fear\_9 &  &  &  &  &  & -0.017 \\
$R^2$ & 0.724 & 0.777 & 0.816 & 0.820 & 0.824 & 0.832 \\
Adjusted $R^2$ & 0.717 & 0.767 & 0.803 & 0.803 & 0.807 & 0.811 \\
P-value(F) & 8.10E-108 & 4.80E-117 & 8.40E-125 & 9.70E-119 & 1.20E-120 & 8.90E-117 \\
Durbin-Watson & 1.854 & 1.958 & 1.972 & 1.998 & 1.922 & 1.883 \\* \bottomrule
\end{longtable}
\emph{Bold values represent statistically significant model coefficients}

\begin{table}[H]
\centering
\begin{tabular}{@{}cr@{}}
\toprule
\textbf{Causality} & \multicolumn{1}{c}{\textbf{P-Value(F)}} \\ \midrule
Death Sentiment $\rightarrow$ Deaths & 4.8E-117 \\
Deaths $\rightarrow$ Death Sentiment & 1.4E-101 \\
\textbf{Conclusion} & \multicolumn{1}{c}{bidirectional causality} \\
Death Sentiment $\rightarrow$ Virus Sentiment & 8.40E-125 \\
Virus Sentiment $\rightarrow$ Death Sentiment & 4.2E-130 \\
\textbf{Conclusion} & \multicolumn{1}{c}{bidirectional causality} \\
Death Sentiment $\rightarrow$ Cases & 9.70E-119 \\
Cases $\rightarrow$ Death Sentiment & 2.09E-46 \\
\textbf{Conclusion} & \multicolumn{1}{c}{bidirectional causality} \\
Death Sentiment $\rightarrow$ Negative & 1.2E-120 \\
Negative $\rightarrow$ Death Sentiment & 8.6E-51 \\
\textbf{Conclusion} & \multicolumn{1}{c}{bidirectional causality} \\
Death Sentiment $\rightarrow$ Fear & 8.9E-117 \\
Fear $\rightarrow$ Death Sentiment & 7.68E-56 \\
\textbf{Conclusion} & \multicolumn{1}{c}{bidirectional causality} \\ \bottomrule
\end{tabular}
\caption{Model 1 Granger Causality}
\label{tab:model 1 causality}
\end{table}

\subsection{Virus Sentiment Model}

With this second model, the ADF test failed for the variable representing the number of daily COVID-19 cases. The cointegration test also failed so that variable was converted to its first-order difference. Table \ref{tab:model 2 lag} displays the values obtained when performing lag selection. A lag of 9 was used for this model as suggested by both the AIC and HQC values. The BIC measure suggested a lag of 8 which is just one off from the selected lag of 9.

\begin{table}[H]
\centering
\begin{tabular}{@{}llll@{}}
\toprule
\multicolumn{1}{c}{\textbf{lags}} & \multicolumn{1}{c}{\textbf{AIC}} & \multicolumn{1}{c}{\textbf{BIC}} & \multicolumn{1}{c}{\textbf{HQC}} \\ \midrule
1 & 11.07 & 11.27 & 11.15 \\
2 & 10.88 & 11.24 & 11.02 \\
3 & 10.65 & 11.17 & 10.86 \\
4 & 10.50 & 11.19 & 10.77 \\
5 & 10.35 & 11.20 & 10.69 \\
6 & 10.11 & 11.12 & 10.51 \\
7 & 9.74 & 10.90 & 10.20 \\
8 & 9.55 & \textbf{10.88*} & 10.08 \\
9 & \textbf{9.41*} & 10.90 & \textbf{10.00*} \\
10 & 9.44 & 11.09 & 10.09 \\ \bottomrule
\end{tabular}
\caption{Model 2 lag selection}
\label{tab:model 2 lag}
\end{table}


Table \ref{tab:model 2 var} shows the obtained model coefficients by regressing each variable against virus sentiment. Virus sentiment is found to have both a positive and negative relationship with lagged values of itself on all lag values (1 to 9), ranging from -0.186 to 0.670. The addition of the daily recorded COVID-19 cases variable was not found to improve the overall model (as seen by a decrease in the adjusted $R^2$ of 0.039) so it was not included in the following models.  The addition of negative sentiment was found significant with two positive and three negative coefficients, their combined coefficients sum up to a total of -0.263. Finally the addition of daily recorded deaths was found to have a very small relationship, both positive and negative on lags 4, 6 and 7 in the order of $10^{-6}$.

On this model, the Durbin-Watson statistics is always found to be slightly over 2 which is within the range which indicates no autocorrelation.

Finally Table \ref{tab:model 2 causality} shows the results from the Granger causality tests. Once again, the observed P values are found to be extremely small and show again a bidirectional causality for all variables.

\begin{longtable}[c]{@{}lrrrr@{}}
\caption{Model 2 VAR Analysis}
\label{tab:model 2 var}\\
\toprule
\multicolumn{1}{c}{\textbf{}} & \multicolumn{1}{c}{\textbf{3.1}} & \multicolumn{1}{c}{\textbf{3.2}} & \multicolumn{1}{c}{\textbf{3.3}} & \multicolumn{1}{c}{\textbf{3.4}} \\* \midrule
\endfirsthead
%
\multicolumn{5}{c}%
{{\bfseries Table \thetable\ continued from previous page}} \\
\toprule
\multicolumn{1}{c}{\textbf{}} & \multicolumn{1}{c}{\textbf{3.1}} & \multicolumn{1}{c}{\textbf{3.2}} & \multicolumn{1}{c}{\textbf{3.3}} & \multicolumn{1}{c}{\textbf{3.4}} \\* \midrule
\endhead
%
\bottomrule
\endfoot
%
\endlastfoot
%
const & \textbf{0.002} & \textbf{0.002} & \textbf{0.007} & \textbf{0.009} \\
VirusDict\_1 & \textbf{0.670} & \textbf{0.608} & \textbf{0.652} & \textbf{0.626} \\
VirusDict\_2 & \textbf{-0.126} & -0.027 & -0.027 & -0.009 \\
VirusDict\_3 & \textbf{0.290} & \textbf{0.186} & \textbf{0.248} & \textbf{0.270} \\
VirusDict\_4 & \textbf{-0.104} & \textbf{-0.098} & \textbf{-0.057} & -0.041 \\
VirusDict\_5 & \textbf{0.140} & \textbf{0.126} & 0.046 & 0.042 \\
VirusDict\_6 & \textbf{-0.186} & \textbf{-0.135} & \textbf{-0.218} & \textbf{-0.243} \\
VirusDict\_7 & \textbf{0.292} & \textbf{0.282} & \textbf{0.257} & \textbf{0.231} \\
VirusDict\_8 & \textbf{-0.163} & \textbf{-0.119} & 0.000 & 0.016 \\
VirusDict\_9 & \textbf{0.072} & \textbf{0.049} & 0.025 & 0.042 \\
Cases\_1 &  & 1.72E-09 &  &  \\
Cases\_2 &  & \textbf{3.47E-08} &  &  \\
Cases\_3 &  & 2.59E-08 &  &  \\
Cases\_4 &  & 1.86E-08 &  &  \\
Cases\_5 &  & -2.09E-09 &  &  \\
Cases\_6 &  & -3.21E-09 &  &  \\
Cases\_7 &  & 9.61E-09 &  &  \\
Cases\_8 &  & 4.67E-09 &  &  \\
Cases\_9 &  & -6.35E-09 &  &  \\
Neg\_1 &  &  & -0.057 & \textbf{-0.072} \\
Neg\_2 &  &  & 0.035 & 0.029 \\
Neg\_3 &  &  & \textbf{-0.176} & \textbf{-0.197} \\
Neg\_4 &  &  & \textbf{0.215} & \textbf{0.206} \\
Neg\_5 &  &  & -0.038 & -0.025 \\
Neg\_6 &  &  & 0.028 & 0.032 \\
Neg\_7 &  &  & \textbf{-0.251} & \textbf{-0.258} \\
Neg\_8 &  &  & 0.030 & 0.017 \\
Neg\_9 &  &  & \textbf{0.059} & \textbf{0.058} \\
Deaths\_1 &  &  &  & 1.81E-06 \\
Deaths\_2 &  &  &  & 1.07E-06 \\
Deaths\_3 &  &  &  & -1.15E-07 \\
Deaths\_4 &  &  &  & \textbf{4.29E-06} \\
Deaths\_5 &  &  &  & -3.39E-07 \\
Deaths\_6 &  &  &  & \textbf{-2.53E-06} \\
Deaths\_7 &  &  &  & \textbf{-2.47E-06} \\
Deaths\_8 &  &  &  & -1.12E-06 \\
Deaths\_9 &  &  &  & -8.55E-08 \\
$R^2$ & 0.812 & 0.779 & 0.871 & 0.881 \\
Adjusted $R^2$ & 0.808 & 0.769 & 0.865 & 0.873 \\
P-value(F) & 1.0E-141 & 4.80E-118 & 2.40E-164 & 1.50E-161 \\
Durbin-Watson & 2.167 & 2.005 & 2.022 & 2.048 \\* \bottomrule
\end{longtable}
\emph{Bold model coefficient values are statistically significant.}



\begin{table}[H]
\centering
\begin{tabular}{@{}cl@{}}
\toprule
\textbf{Causality} & \multicolumn{1}{c}{\textbf{P-Value(F)}} \\ \midrule
Virus Sentiment $\rightarrow$ Cases & 4.80E-118 \\
Cases $\rightarrow$ Virus Sentiment & 3.91E-42 \\
\textbf{Conclusion} & \multicolumn{1}{r}{bidirectional causality} \\
Virus Sentiment $\rightarrow$ Negative & 2.40E-164 \\
Negative $\rightarrow$ Virus Sentiment & 3.56E-61 \\
\textbf{Conclusion} & \multicolumn{1}{r}{bidirectional causality} \\
Virus Sentiment $\rightarrow$ Deaths & 1.50E-161 \\
Deaths $\rightarrow$ Virus Sentiment & 2.40E-217 \\
\textbf{Conclusion} & \multicolumn{1}{r}{bidirectional causality} \\
\bottomrule
\end{tabular}
\caption{Model 2 Granger Causality}
\label{tab:model 2 causality}
\end{table}

\subsection{Positive Model}

This final model attempts to find relationships for a positive sentiment. The ADF tests failed on the vaccine sentiment variable as well as the doses variable which represents the amount of daily injected vaccine doses. The cointegration test also failed on these variables so as a results both variables were differentiated to their first-order. Table \ref{tab:model 3 lag} shows the results for lag selection. A lag length of 8 was selected for this model as it was selected best by two of the three indicators (AIC and HQC). The BIC metric suggested a lag of 6.

\begin{table}[H]
\centering
\begin{tabular}{@{}llll@{}}
\toprule
\multicolumn{1}{c}{\textbf{lags}} & \multicolumn{1}{c}{\textbf{AIC}} & \multicolumn{1}{c}{\textbf{BIC}} & \multicolumn{1}{c}{\textbf{HQC}} \\ \midrule
1 & -5.89 & -5.69 & -5.81 \\
2 & -5.98 & -5.63 & -5.84 \\
3 & -6.12 & -5.62 & -5.92 \\
4 & -6.26 & -5.60 & -6.00 \\
5 & -6.74 & -5.92 & -6.41 \\
6 & -6.97 & \textbf{-6.01*} & -6.59 \\
7 & -7.06 & -5.93 & -6.61 \\
8 & \textbf{-7.16*} & -5.88 & \textbf{-6.65*} \\
9 & -7.11 & -5.68 & -6.55 \\
10 & -7.14 & -5.55 & -6.51 \\ \bottomrule
\end{tabular}
\caption{Model 3 Lag Selection}
\label{tab:model 3 lag}
\end{table}
 
A VAR model was built incrementally by adding variables one at a time. Table \ref{tab:model 3 var} shows the coefficients as well as key metrics obtained by regressing each of the variables against the positive sentiment variable. Each variable added to this model improved the overall model as shown by a constantly growing adjusted $R^2$ measure, growing from 0.204 to 0.567 on the last iteration. Positivity was found to have a positive relationship on the lagged values of itself at 1, 2, 5, 6 and 8 days with the largest coefficient appear on day 1 (0.314). The vaccination sentiment was added to the model and positive coefficients were found in days 3 to 7 with coefficients ranging from 1.014 to 1.839. Joy was found both to have a positive and negative relationship on days 2 to 6 where the coefficients are found to roughly cancel each other (their sum gives a result of about -0.22). Finally, the vaccination daily count variable was added to the model finding a very small negative relationship on day 7 in the order of $10^{-8}$.

For this table the Durbin-Watson statistic was consistently below 2, ranging between 1.986 and 1.891 which is a sign of small positive autocorrelation.

A Granger causality test was also performed on this model and results are shown in Table \ref{tab:model 3 causality}. Like the other two models, the observed P-values for the Granger causality tests were found to be extremely small, thus showing bidirectional causality for all of the introduced variables of this model.

\begin{longtable}[c]{@{}lrrrr@{}}
\caption{Model 3 VAR Analysis}
\label{tab:model 3 var}\\
\toprule
\multicolumn{1}{c}{\textbf{}} & \multicolumn{1}{c}{\textbf{3.1}} & \multicolumn{1}{c}{\textbf{3.2}} & \multicolumn{1}{c}{\textbf{3.3}} & \multicolumn{1}{c}{\textbf{3.4}} \\* \midrule
\endfirsthead
%
\multicolumn{5}{c}%
{{\bfseries Table \thetable\ continued from previous page}} \\
\toprule
\multicolumn{1}{c}{\textbf{}} & \multicolumn{1}{c}{\textbf{3.1}} & \multicolumn{1}{c}{\textbf{3.2}} & \multicolumn{1}{c}{\textbf{3.3}} & \multicolumn{1}{c}{\textbf{3.4}} \\* \midrule
\endhead
%
\bottomrule
\endfoot
%
\endlastfoot
%
const & \textbf{0.055} & \textbf{0.053} & \textbf{0.041} & \textbf{0.039} \\
Pos\_1 & \textbf{0.380} & \textbf{0.371} & \textbf{0.335} & \textbf{0.314} \\
Pos\_2 & 0.069 & 0.032 & \textbf{0.105} & \textbf{0.120} \\
Pos\_3 & -0.052 & -0.012 & -0.040 & -0.031 \\
Pos\_4 & \textbf{0.149} & \textbf{0.099} & 0.080 & 0.082 \\
Pos\_5 & -0.034 & 0.018 & \textbf{0.078} & \textbf{0.091} \\
Pos\_6 & 0.013 & 0.008 & 0.058 & 0.063 \\
Pos\_7 & \textbf{0.093} & \textbf{0.134} & \textbf{0.138} & \textbf{0.137} \\
Pos\_8 & -0.043 & -0.058 & \textbf{-0.090} & \textbf{-0.090} \\
VaccineDict\_1 &  & 0.721 & 0.320 & 0.321 \\
VaccineDict\_2 &  & -0.264 & -0.479 & -0.385 \\
VaccineDict\_3 &  & \textbf{2.086} & \textbf{1.648} & \textbf{1.839} \\
VaccineDict\_4 &  & \textbf{1.088} & \textbf{1.185} & \textbf{1.238} \\
VaccineDict\_5 &  & \textbf{2.015} & \textbf{1.261} & \textbf{1.369} \\
VaccineDict\_6 &  & \textbf{1.906} & \textbf{0.918} & \textbf{1.141} \\
VaccineDict\_7 &  & \textbf{3.481} & \textbf{0.989} & \textbf{1.014} \\
VaccineDict\_8 &  & \textbf{1.693} & 0.410 & 0.380 \\
Joy\_1 &  &  & -0.408 & -0.313 \\
Joy\_2 &  &  & \textbf{2.238} & \textbf{2.187} \\
Joy\_3 &  &  & \textbf{1.334} & \textbf{1.343} \\
Joy\_4 &  &  & \textbf{2.123} & \textbf{2.203} \\
Joy\_5 &  &  & \textbf{-2.920} & \textbf{-2.908} \\
Joy\_6 &  &  & \textbf{-2.094} & \textbf{-2.249} \\
Joy\_7 &  &  & 0.425 & 0.286 \\
Joy\_8 &  &  & -0.348 & -0.345 \\
VaccineDoses\_1 &  &  &  & -1.01E-09 \\
VaccineDoses\_2 &  &  &  & -4.26E-08 \\
VaccineDoses\_3 &  &  &  & -1.02E-08 \\
VaccineDoses\_4 &  &  &  & -1.72E-08 \\
VaccineDoses\_5 &  &  &  & 1.94E-08 \\
VaccineDoses\_6 &  &  &  & -3.20E-08 \\
VaccineDoses\_7 &  &  &  & \textbf{-9.08E-08} \\
VaccineDoses\_8 &  &  &  & -5.59E-09 \\
$R^2$ & 0.204 & 0.338 & 0.548 & 0.567 \\
Adjusted $R^2$ & 0.188 & 0.312 & 0.520 & 0.531 \\
P-value(F) & 8.12E-17 & 2.32E-27 & 1.96E-53 & 7.17E-52 \\
Durbin-Watson & 1.986 & 1.913 & 1.891 & 1.904 \\* \bottomrule
\end{longtable}
\emph{Bold model coefficient values are statistically significant.}



\begin{table}[H]
\centering
\begin{tabular}{@{}cr@{}}
\toprule
\textbf{Causality} & \multicolumn{1}{c}{\textbf{P-Value(F)}} \\ \midrule
Positive   $\rightarrow$ Vaccine Dict & 2.32E-27 \\
Vaccine Dict $\rightarrow$  Positive & 1.17E-15 \\
\textbf{Conclusion} & bidirectional causality \\
Positive   $\rightarrow$  Joy & 2.32E-27 \\
Joy $\rightarrow$  Positive & 1.92E-35 \\
\textbf{Conclusion} & bidirectional causality \\
Positive   $\rightarrow$  Vaccinations & 7.17E-52 \\
Vaccination $\rightarrow$  Positive & 7.18E-37 \\
\textbf{Conclusion} & bidirectional causality \\
\bottomrule
\end{tabular}
\caption{Model 3 Granger Causality}
\label{tab:model 3 causality}
\end{table}

