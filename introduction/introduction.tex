\chapter{Introduction}\label{Introduction}

This chapter will begin with an introduction to the motivation which led to this research project. It will be followed by a section proposing the research question then by the objectives that have been set to best address this question. Finally, this chapter will conclude with a structure to the remainder of the thesis.

\section{Motivation}\label{Motivation}
Looking back in history, it is clear that the world went and will go through many phases
whether it be a pandemic, a war or a revolution to name a few. What makes the recent COVID-19 pandemic particularly interesting is that it is the first to be spread throughout the world during the digital age, creating new behaviours on the web.   
In the context of a global pandemic in a digital age, the role of news agencies is more important than ever, with the world population in quarantine all looking for the same information, seeking to understand the situation they are in.

One could manually read newspaper articles from many different sources about a topic in an attempt to understand how information is conveyed to the population, however such a study would be limited in the manpower available and would require a strict, well defined approach to do so. Taking such an approach is thus limited in its nature.

Natural Language Processing (NLP) is a sub-field of computer science which enables analysis of written text, Sentiment Analysis makes use of NLP in order to extract subjective information from texts. Examples of this include extracting and quantifying the polarity or certain emotions from text.
To this day, most of the studies performed and the applications of Sentiment Analysis have been focused on the English language, leaving space for many new studies, specifically in languages that are constructed significantly differently such as French.

The French media publishes vast amounts of news articles daily, all of which can be used to extract meaningful information such as to ultimately understand hidden patterns in how information is communicated to the population in the context of a global emergency that is the COVID-19 pandemic.
This thesis sets out to attempt to automate the process of discovering such hidden patterns through the use of sentiment analysis on french news.

\section{Research Question}\label{Research Question}

The research question for this thesis can be summarized as follows:

\begin{center}
    \emph{To what extent does the sentiment of French news media relate to France's pandemic metrics?}
\end{center}

\section{Research Objectives}\label{Research Objectives}
To structure the work done and address the research question effectively, the following key objectives were defined:
\begin{enumerate}
    \item Gather a large corpus containing high quality French news articles relevant to COVID-19
    \item Create specialised dictionaries targeted to measure pandemic-related sentiments
    \item Perform sentiment analysis on the corpus to retrieve measures for various sentiments and term frequencies
    \item Conduct statistical analysis on the results in order to identify and quantify potential relationships
\end{enumerate}

\section{Report Structure}\label{Report Structure}

The next chapter will start with a look into the current literature on the role of the media in the context of a global emergency such as the COVID-19 pandemic (Chapter \ref{Literature Review}). This is followed by a review and discussion of the current literature for sentiment analysis, which will cover the main approaches used for its various applications. This chapter concludes with a discourse of the statistical methods used throughout this report.

The literature review is followed by a section on the method used (Chapter \ref{Method}). The method includes details of the approach taken to solve the research objectives previously laid out. It is split into three parts: data gathering which focuses on constructing and collecting data for the analysis, sentiment analysis which details of the actions performed on the corpus to measure various sentiments and finally a section on the statistical analysis performed.

The following section (Chapter \ref{Evalutation}) discusses the results obtained from applying the method as previously described. It will begin with a review of the built corpus and include some initial analysis. Following will be the results of the corpus through the use of sentiment analysis and various dictionaries, some of which were built specifically for this study. This chapter will close with a discussion of the statistical analysis performed on the obtained metrics.

Finally, this report concludes with a review of the method which was used to perform this study (Chapter \ref{Conclusion}). Then a commentary on the key findings and the possible improvements will be carried out which outlines directions for future work. The thesis will come to an end with final comments on the project, providing insights into the challenges that were encountered and a brief reflection on the personal achievements that were made throughout this project.
